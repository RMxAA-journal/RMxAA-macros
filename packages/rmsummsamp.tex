\documentclass[proceedings]{rmaa}
\title{Conference Summary}
\author{Donald P. Cox
  \affil{Department of Physics, University of Wisconsin-Madison, USA} }

\fulladdresses{
\item Donald P. Cox: Department of Physics, 
University of Wisconsin-Madison, 1150 University Ave.,
Madison, WI 53706, USA (cox@wisp.physics.wisc.edu).}

\shortauthor{Cox}
\shorttitle{Conference Summary}

%\keywords{}
%\abstract{}
%\resumen{}
%% Indexing commands
\listofauthors{D.~P.~Cox}
\indexauthor{Cox, D.~P.}
%% End of Indexing commands

\begin{document}

\maketitle

%Note: sybols meant as "bullets" may not come through properly, also ' and ".
%Don


\noindent The Plasma 99 conference, on Codes, Models, and Observations, appeared at 
first sight to have had a widely dispersed scientific content, making a 
summary almost impossible, at least for me.  Feeling rather inadequate to 
the task at the beginning of the meeting, I polled the attendees via a 
show of hands and found that only seven would admit to having been asked, 
before me, to present the summary (and had had the good sense to say 
no).  That being a rather small number I decided that the organizers had  
had a sufficient confidence in me and I rededicated myself to the honor 
of this task.  (Sadly, Pepe Franco was busy with organizational details 
at that moment and missed the joke entirely.)  Fortunately, by the end of 
the week I was able to discern the genius of the organizers, and came to  
a form of synthesis that I at least could enjoy.

My oral summary began with a list of people who had particularly
contributed to the successful operation of the meeting and an
expression of gratitude to the Local and Scientific Organizing
Committees.  Special thanks went to Jane Arthur, Luc Binette, Marco
Martos, Julia Espresate, Lorena Arias, Anabel Arrieta, Sandra Ayala,
Wilder Chicana, Eduardo de la Fuente, Fulgencio Garc\'{\i}a, Luis
L\'opez, Alberto Nigoche, Eva Noyola, Carlos Rodr\'{\i}guez, Juan
Segura, and Lucero Uscanga.  I managed to elicit hugs from Jane Arthur
and later Julia Espresate in the process of garnering a general
acclamation for this tireless group.  In retrospect, I should have
remembered to thank Pepe Franco and the Instituto, and figured out
some way to get a hug from Sylvia Torres-Peimbert as well, but I was
nervous about what was about to happen next and ungraciously forgot.
So, let me here record said gratitude and apologize for not having
remembered at the time to let the rest of the participants express
theirs as well.  (When I once before failed in such a mission, Carl
Heiles jumped up to save the day.  Unfortunately, this time he was not
present.)  But those who were there will likely understand my
nervousness and possibly be a little forgiving.

In order to bring to the attention of the participants at the begining of 
the week that this was not a meeting within specialties, but one which 
transcended them, I had made a request that during the week each 
participant provide me with a note about something they found 
interesting, or a question they would like to see discussed in the 
summary session, one which specifically involved an area outside their 
own personal sphere of expertise.  I thought in this way that I might be  
guided to topics of broad interest for the closing discussion.  I 
received several such notes during the week, and quite a few more on 
Friday, but by then had decided to begin with a different sort of 
synthesis, one which arose from the many interesting conversations I had  
had during the week.  That synthesis involved several themes, the first 
of which was
\vspace*{-0.5ex}

\section{Dancing}
\vspace*{-0.5ex}

Following the wonderful banquet in Hotel Cortez, it turned out that the 
organizers had made arrangements for us to go to a club where very 
enjoyable dance music was featured.  Not everyone went.  I did, but found 
that it was a very good time to switch to 7-Up.  The dance went on for a  
long time and was spectacularly ecumenical.  Toward the end I found 
myself seated near Alberto L\'opez as we contemplated youth, beauty, and 
spirit.  He ventured the opinion that a photograph of the dance floor at  
that moment would be the perfect conference summary.  I believed I knew 
what he meant, and agreed at the time.  But by the next morning, or 
rather, four hours later when it was time to get up for the last day of 
the meeting, my opinion had drifted some.  It is true that the spirit we  
witnessed, participated in, and drank fully from was an important aspect  
for us, but not everyone was there dancing, nor should all have been.

        I myself found a home in astronomy.  I first knew it at the IAU 
meeting in Australia in 1973 where I found the form of family I had 
waited for all my life.  But the members weren't all the same.   Some 
dance and carrouse, some are retiring away from the podium.  But we are a 
family who needs to understand, and I believe we largely do, that a warm  
spirit for all of our diverse members is a big part of who we are.  It is 
one thing about our business in which I have always been proud.  I don't  
know Alberto L\'opez very well, had not met him before this conference, but 
I felt that warmth very much from him, right away.
\vspace*{-0.5ex}

\section{Information Overload}
\vspace*{-0.5ex}

The second theme was less stirring to the assembled participants, the 
subject having to do with how we individually cope with the flood of 
information passing through our lives.  Having been around when such 
information was largely nonexistent, and having become rather comfortable 
with that circumstance, perhaps I am personally more shaken than the 
younger people for whom the present day is more the norm.  In any case, 
my attempt to elicit survival techniques from the group led to no useful  
insights, so we went on to the next theme.  (I suspect that the answer, 
though, is widespread collaboration.)
\vspace*{-0.5ex}

\section{Enthusiasm}
\vspace*{-0.5ex}

One of the things I felt and heard most strongly, after my own 
trepidation, was enthusiasm for the present high rate of progress.  The 
flood of current activity, as represented in the meeting, was distinctly  
spectacular.  There are several broad fronts which I will summarize next.
\vspace*{-0.5ex}

\subsection{Atomic Data}
\vspace*{-0.5ex}

        Theory, measurement, assemblage, testing against the real world, 
identification of diagnostics, simplification (condensation) into usable  
friendly modeling and analysis tools,\dots, a remarkable renaissance of 
activity is currently flourishing and carrying us forward into entirely 
new times, and all aspects were impressively represented at the conference.
\vspace*{-0.5ex}

\subsection{MHD Codes}
\vspace*{-0.5ex}

        Hydrocodes are everywhere now, but ones which are 3 dimensional, 
which include MHD with divergence free magnetic fields, thermal 
conduction, separate electron and ion temperatures, self gravity, 
radiative transfer, adaptive mesh refinement, sufficient dynamic range 
for resolution of turbulence, evolution of the nonthermal component 
(cosmic rays), etc., and are designed for parallelism, are willingly 
shared and are reasonably well documented are right around the corner, 
with dedicated folks working feverishly to bring them to our tool boxes.   
Many aspects are already in place, with hundreds of users.  More physics  
is better, and the sooner the better.  (1/1/2000, the promised release 
date for the new ZEUS, will occur before this volume reaches print!)  
Again we find ourselves living in times of fantastic potential.
\vspace*{-0.5ex}

\subsection{Acquisition of Astronomical Data}
\vspace*{-0.5ex}

        Instrument conception, design, development, promotion, flight, 
calibration, and support with dedication beyond my imagination has led us 
to a state with ever more amazing acquisition of images and spectra, 
revealing enormous detail, sometimes with sufficient detail that we are 
able to decode it via:
\vspace*{-0.5ex}

\subsection{Comparisons With Detailed Modeling}
\vspace*{-0.5ex}

        Combining tools from the atomic data projects and MHD 
development, to produce synthetic images and spectra of specific 
evolutionary scenarios is providing incredibly detailed insight into the  
nature of the beasts on which recent data has been taken.  And this is 
just the beginning, with several new observational tools up and running 
and others in the wings, some ready to take the stage even in the next 
few months.  We are most definitely moving beyond the age of the 
spherical cow.
\vspace*{-0.5ex}

\section{Pause for an Observation of My Own}
\vspace*{-0.5ex}

It is my perception that a lot of the folks developing tools for all of 
us seem to be happier doing astrophysics, applying the tools.  Why is that?

        In order to avoid putting anyone on the spot, I will deflect my 
example to a nonconference situation.  Just this morning I was describing 
this observation to a friend in my office when a rather famous and 
relatively young astronomer was standing outside.  Overhearing my remark, 
the external observer immediately responded with considerable passion 
(Wisconsin is, after all, a passionate land).  I don't remember his exact 
remark, but it was to the effect that in application was recognition.

        There are a lot of reasons why application is seductive.  It is 
more fun to talk about with your friends.  It gets papers written, 
bringing both recognition and salary responses.  For a lot of people, it  
is just plain more fun, more exciting, more interesting, for whatever 
reasons.  Yet they spend major portions of their lives in design, 
construction, and implementation of our tools---whether they be atomic 
data tools, plasma code tools, MHD modeling tools, instruments of 
observation, or the software tools making the observational data accessible.

        And we often forget to express our gratitude, whether it be 
verbally, out of basic human recognition for their efforts and sacrifice, 
or through lobbying for greater recognition of and support for their 
endeavors.

        This meeting was unusually alive with such people, and it was 
wonderful to have a chance to see their wares, and express our pleasure 
at their accomplishments.  

        It was my hope that in highlighting this phenomenon, many more of 
us would see the need to be more appreciative of these efforts, both for  
a moment during which we had a round of applause (though there were so 
many to be on the receiving end that I wondered who besides me would be 
left to clap), and as a general theme to carry along into our collective  
future.

A word now from the (slightly) old(er) guy.  Absolutely everyone feels 
underappreciated at least part of the time.  Some of us occasionally get  
the opportunity to feel overappreciated.  The former makes a person 
crabby and sad; the latter makes one feel hollow and fraudulent.  It's 
better just to know when you do good work, to have a few friends who know 
it too, and not to worry about what others think beyond that.  Oh yes, 
and not to forget this, that part of your job is to remember to be 
appropriately appreciative of others.
\vspace*{-0.5ex}

\section{Relevant to that Last Remark, a Brief Aside: Taking Baby Steps}
\vspace*{-0.5ex}

We try to understand how those great mysteries in the sky do what they 
appear to.  Usually we start off with very simple thoughts on the 
matter.  Several of us might try at the same time, with different 
thoughts.  We might get into arguments about who is right.  Generally, 
neither is.  One has overlooked this, another neglected that.  It is my 
conviction that, in the end, truth is approached via diffusion, and even  
then only after the observational and analysis tools have improved to the 
point that there is almost no room left for excessive naivete.

        This situation leads me to several admonitions, involving 
criticism, listening, and patience.

\paragraph{Criticism.}  It is my conviction that when you see one of our 
compatriots commiting errors of neglect, you have an obligation to tell 
them so, privately, so that if you are correct and persuasive, they have  
a chance to do better in the next round.  But try not to put them on the  
defensive.  Recognize that it is all part of the great diffusion process, 
and that sustaining openness of minds while biasing the diffusion in a 
useful direction is your primary goal.
\paragraph{Listening.}  Pay attention to the data.  If it begins to turn against  
you, it is better to be the first to notice, and humbly readjust yourself 
to a new perspective.  This is easier if you know in advance that your 
toy models are not likely correct in the first place.  They are attempts  
to understand, not unimpeachable truths.
\paragraph{Patience.}  A great deal of patience is needed (though not too much or 
nothing gets done).  It takes a long time to improve and promulgate your  
ideas, even your tools, as handy as they are.  You and your wares are not 
likely to be right, accepted, or appreciated at first, but you can at 
least be interesting.  Tell your friends.  Get their criticisms and 
garner their enthusiasms.  Even better, if you can find good enemies, get 
their criticisms.  Listen.  Improve.  Keep being interesting.  Someday, 
you might even be right, or your products ready.  Even then it will be 
five years before anyone notices, because the rest of us are too busy 
with our own development.
\vspace*{-0.5ex}

\section{Theorists}
\vspace*{-0.5ex}
\enlargethispage{2ex}
Almost nothing we know of in the universe was predicted before being 
seen.  Most ``theory'' is attempted interpretation, twisting our brains and 
tools to solve a puzzle.  It's no different really than trying to figure  
out how to measure the spectrum of soft X-rays from the diffuse 
interstellar medium, except in the latter case a detector and payload 
have to be built, the whole thing set atop a slowly exploding bomb and 
shot into the sky for a brief but exciting ride, the data collected, the  
instrument calibrated, the unexpected bugs deciphered, and another shot 
planned to get it right the next time, and so on.  People tell me that 
theorists are somehow more highly regarded than experimentalists.  For 
the love of God, I've never been able to understand why. 
\vspace*{-0.75ex}

\section{Images versus Spectra}
\vspace*{-0.75ex}

We heard the usual, ``{\it A spectrum is worth a thousand pictures}'', extended 
by Deborah Dultzin-Hacyan to ``{\it A thousand spectra are worth more than one  
average spectrum}''.  After that she showed that from many spectra one 
could get hints of the image, an image that in her case was totally 
unavailable because of the extremely small scale being sampled, somewhat  
turning things around, you might say.  One could hear a yearning within 
the cleverness, for the picture that could never be had.  In Madison, 
Blair Savage is often heard extolling the virtues of spectra.  I have 
never totally agreed with him, but until now could not put my finger on 
just what bothered me.
  
        At the conference we saw many wonderful images, and then MHD 
models of systems which closely resembled those images.  We saw many 
wonderful spectra, and the kinds of detailed insight that could be mined  
from those spectra.  I reflected on this, along with the conviction 
expressed above that almost nothing we know of in the universe was 
predicted before being seen.  I think I finally have it right.  We need 
both, for different reasons:\vspace*{1ex}

\noindent\hspace*{6em}1. IMAGES  are desperately needed for the generation of ideas.  \\
\hspace*{6em}2. SPECTRA  are desperately needed for constraint of that imagination.

In either case, higher resolution brings greater access to truth.
\vspace*{-0.75ex}

\section{More on the Mixed, Exciting, Nurturing Sociology}
\vspace*{-0.75ex}

When I was young(er), I would go to meetings and, apart from giving my 
own presentation, I wondered about what I was there to accomplish.  Of 
course I was anxious to learn from what others had to say in and after 
their talks, but I pondered the sociological aspects of the situation.  
To some extent, I imagined that part of my job was to impress the old 
guys.  Truth be known, I never considered it to be a very substantial 
part---the old guys always had lots of other folks clammoring for their 
attention and I never was very inclined to be a valence electron for a 
Z=18 atom, an old guy being the nucleus.  But I always felt their 
presence to some extent, appreciated their questions, and felt good when  
their attentions shone on me, however briefly.  Then one day they weren't 
there any more.  I felt their absence.  

        The good news is that over the years I had spent most of my time 
and attentions on my cohort, the strange and exciting lot of folks 
roughly my own age.  And they were still there, though suddenly we were 
the old guys, though not all guys.  It was a considerable shock; I had 
simply never considered the possibility that I would outlive my heroes, 
thereby becoming an elder.  

        What was my new role, I wondered.  Abdication sounded good to me, 
and I began associating more and more with younger and younger people, 
who after all are our future, and for the most part our present as well.

        Omigosh, I just remembered something!  Spitzer's book on Physical 
Processes in the ISM has the following dedication, ``{\it To the younger 
generation from whom I have learned so much}''.

        Heroes, I guess, have blazed trails on more than one mountain.  
What have they shown us to be useful in this case?  Perhaps our role is 
to try to help those younger people focus their efforts in productive 
ways and to stand back a bit from the immediate problem or calculation to 
ask whether the current result makes sense and fits within the larger 
picture.  It involves a bit more than just dancing.
\vspace*{-0.5ex}

\section{And What About Science?}
\vspace*{-0.5ex}
\enlargethispage{2ex}
        There were, in the end, about fourteen suggestions submitted by 
the participants for discussion or highlighting, in addition to several 
more that came to me verbally or through my own impressions.  Not all 
actually got discussed, but the list itself provides a nice perspective 
on the nature of the meeting and the souvenirs carried away by the 
participants.  In no particular order, the questions and notions are 
listed below.
\begin{itemize}
\item  What are the best bets for the origin of temperature variations in 
gaseous nebulae, variations needed to match model spectra with observations?

\item  Why do the models of various systems so often seem to get the [\protect\ion{O}{1}]  
6300 Angstrom line strength wrong?  Are we missing something fundamental?

\item  Isn't it finally time to abandon the fiction of filling factors and  
model the complexity we see?  How do we do that?

\item  Could the asymmetries observed in the very faint external halos of 
many large multiple shell planetary nebulae, attributed to their 
interaction with the ISM, be used to probe the density distribution of 
the interstellar medium?  Is the apparent fact that 60\% show no asymmetry 
an indication that a similar fraction of the ISM is at very low density,  
or is it primarily due to the difficulty of seeing the faint haloes, 
compounded by projection effects?  Or are they just not moving through 
the ISM fast enough to be distorted (5 to 10 km~s$^{-1}$ appeared to give 
noticeable distortions in the models.)?  

\item  If, owing to the general insignificance of thermal pressure anyway,  
large regions of the ISM could be largely vacant (except for magnetic 
field and cosmic rays), what would that be like?  Vacant old bubbles?  
Vacant flux bundles?  What?

\item  If the reionization of the universe was as patchy as the models 
suggest, could there be places that were missed?

\item  The simulation shown by Dinshaw Balsara appears to vindicate old dogma 
(astronomical bodies can run dynamos) and violate current dogma (dynamos  
get gummed up at small scales), indicating that a mean field dynamo is a  
real possibility.  As the calculation resolves the scales within which it 
was expected to fail, the new dogma must have some holes in it.  This is  
very good news for those of us who hope to understand astrophysical 
magnetic fields.

\item  Could the interesting discussion of rapid reconnection lead to an 
understanding of the power law distribution of solar flares?  One 
participant replied, ``I think the answer is yes.  If turbulence is a 
critical component of reconnection, then since the release of energy from 
reconnection helps  drive reconnection in adjacent regions, and one 
expects the natural emergence of a power law distribution of flares.  
It's like the old sandpile problem.''

\item  The sobering presentations concerning the inadequacy of our plasma 
emission codes, and the enormous effort presently underway to rectify 
that, tying our understanding to laboratory measurements to the extent 
possible was reflected upon by several participants, with the word 
``exciting'' being common.  One was concerned that the developing tools be  
optimized for application to relatively weak X-ray sources for which 
nothing better than CCD resolution may be available for quite some time.

\item  Several people expressed concern, though not with regard to things 
heard at the meeting, that spectral fitting is often inappropriate, that  
perhaps some of the analysis tools for X-ray spectra have become too 
convenient, with people relying too much on them rather than using a more 
holistic modeling scheme.  It may be a good example of diffusing toward 
understanding.

\item  What is magnetic helicity anyway?  Why does it matter?

\item  The fact that the mean quasar spectrum closely resembles the sum of  
all photoionization models was striking; the selection effects for line 
strengths are so strong that a wide range of conditions always looks 
about the same.  What do we have to do to dig out more fundamental 
truths?  One example of examining details was a study of the time 
variability of the \ion{O}{8} K-shell edge compared to the constant \ion{O}{7} 
absorption in a ``warm absorber'', in which it was concluded that warm 
absorbers are spatially extended multi-zone regions in which different 
parts are absorbing at different times.

\item  One participant noted that the development of ZEUS-MP is a nice 
example of how a community with common interests may gather around a 
project, expressing the hope that such gathering could become more 
common. 

\item  Where are the ``collective'' plasma effects in the interstellar medium 
hiding?  Is it nature or we who are doing the hiding?  (I'm not sure what 
this question is about, but maybe that just makes me one of the hiders.)

\item  The apparent need for heating of the warm ionized gas in the Galaxy  
beyond that available from photoionization stirred interest, the 
underlying assumption being that the ionization itself arises from 
photons.  The distribution of the H$\alpha$ emission shown from the Perseus 
Arm certainly gave the impression of concentration in large loops 
reminiscent of those on the solar surface.  One participant wondered 
whether flare-like activity may have something to do with both the 
ionization and heating.  Another contemplated the ionization advertised 
as being available from the X-ray emitting gas in the galaxy and whether  
in sum with the contribution from stars in the disk might lead to a net 
increase in photoelectric heating further from the plane---but the idea 
turned out to be inconsistent with the data that shows that the increased 
heating needed is correlated with density, not distance from the plane.

\item  Is there any evidence in the large number of Proplyds of the Orion 
Nebula for any evolutionary effects that might shed light on the star 
formation history of the region?

\item  Turbulence is beginning to seem important, for stirring, for enhancing 
transport, for the corresponding possibilities for enhancement of 
reconnection and thermal conduction, for the eventual dissipation and 
heating of the gas.  How close are we to getting it right?

\item  Disks and Jets, Disks and Jets.  Doesn't nature have any other ideas?  
What happened to spheres?

\item In the core-halo observations of ultrcompact \ion{H}{2} regions,
  aren't the core sizes showing us the actual size of the regions
  which form massive stars?

\item  When, exactly, did appealing to magnetic fields become respectable?

\item  Watch out for those resonance lines, they're often optically thick and 
for some geometries their ratios to other lines can fool you.
\end{itemize}

And, of course, I was personally appreciative of the wonderful 
presentations on:
\begin{itemize}
     \item  Ultraviolet and X-ray diagnostics of the solar corona,
     \item  The high $z$ structure of a galactic spiral density wave, 
     \item  Error estimates for X-ray emission lines in the hydrogen and 
helium isosequences,
     \item  Dust and density distributions in \ion{H}{2} regions,
     \item  Photoionization of galactic gas by old supernova remnants, and
     \item  Hydrodynamic simulations of supernova remnants in diffuse 
environments,
\end{itemize}
made by the six of my former graduate students who were at the meeting.
\vspace*{-0.5ex}
\enlargethispage{2ex}
\section{Ending}
\vspace*{-0.5ex}

The meeting closed after another round of applause for the organizers, 
whose genius had by this point had been revealed.  Pepe asked me to 
express particular appreciation to Jane Arthur, and Nancy Brickhouse.  
All participants were aware of many of Jane's apparently tireless 
contributions, while Pepe found Nancy's help also invaluable in laying 
out parts of the scientific program.

% This is necessary since we have no bibliography
\outputfulladdresses

\end{document}


