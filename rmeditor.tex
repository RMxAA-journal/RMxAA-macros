\documentclass[guide]{rmaa}
      
\title{The \RMAA\ \LaTeX\ Macros: A Guide for Editors\altaffilmark{1}}
\author{W. J. Henney\altaffilmark{2}
  \affil{Instituto de Astronom\'{\i}a, UNAM, Morelia} 
  }
\altaffiltext{1}{\protect\raggedright The latest version of the macros 
  should be available from
  \texttt{http://www.astrosmo.unam.mx/rmaa}} 
\altaffiltext{2}{Email: \texttt{will@astrosmo.unam.mx}}
\fulladdresses{
  \item W. J. Henney: Instituto de Astronom\'{\i}a, UNAM, J. J. Tablada
    1006, Lomas de Santa Mar\'{\i}a, 58090 Morelia, M\'exico
    (will@astrosmo.unam.mx)} 
    
\shortauthor{Henney} \shorttitle{\RMAA\ Editor Guide}

\ReceivedDate{2000 Feb 09} 
\AcceptedDate{2000 Feb 09} 
\SetYear{2000}
\resumen{Este documento describe como usar el ``\LaTeX\ document
  class'' \texttt{rmaa.cls} para preparar un n\'umero entero de la
  \emph{\RMAAlong} o \emph{\RMAASClong}.  } 

\abstract{This document describes how to use the \LaTeX\ document
  class \texttt{rmaa.cls} to produce an entire issue of the
  \emph{\RMAAlong} or \emph{\RMAASClong}.}

\keywords{Typesetting: \LaTeX\  --- Journals: Astronomical} 

\makeglossary{} 

\begin{document}
\maketitle
\tableofcontents 

\section{Introduction}
\label{sec:introduction}

This Guide discusses some of the trickier aspects of using the
\texttt{rmaa} class. In particular, how to use it to format an entire
volume of the \RMAA\ or \RMAASC\ by writing a container file that
simply |\input|'s the individual stand-alone article files. With very
little extra effort on the part of the editors, a fully
cross-referenced ``Table of Contents'' and ``Index of Authors'' will
be automatically generated. Commands also exist for producing the
``Preface'', ``Group Photograph'', ``List of Participants'', etc., and
for formatting the abstracts of authors who couldn't manage to write
up their contribution \emph{in extenso} on time. Furthermore, a script
is provided that will then extract the postscript files of the
individual components (articles\footnote{Appending plate pages if
  appropriate.}, index, preface, etc.) and generate an HTML table of
contents that points at them\footnote{The HTML part still needs some
  work.}. The idea is that these can then be linked to from ADS.

\subsection{Intended audience}
\label{sec:intended-audience}

This guide may be of interest to the following:
\begin{itemize}
\item My future self
\item Guest editors of volumes of \RMAASC
\item Editors/editorial assistants of \RMAA
\item Even authors, if they're brave\dots
\end{itemize}


\subsection{Shortcomings of this Guide}
\label{sec:shortcomings}
Many, I'm sure, but most notably: 
\begin{itemize}
\item It is far from complete
\item I assume throughout that you are working on a Unix (or Linux)
  system. 
\end{itemize}

\section{Individual papers}
\label{sec:individual-papers}

Before you try to format the entire book, you should first make sure
that each individual paper is exactly how you want it and can be
\LaTeX ed without producing any errors.  Most of the necessary
information is given in the Guide for Authors, but in this section I
give further technical details.

\subsection{References}
\label{sec:references}

These are a major source of author errors. References to conference
proceedings are particularly troublesome, especially when the
proceedings are part of a series. Try to follow ApJ style, but don't
worry about it too much, even they are inconsistent sometimes.

\subsection{Figures}
\label{sec:figures}

Here I discuss various issues related to figures: how to get them to
come out where you want them, how to curmudgeon misbehaving PostScript
files into submission, and how to deal with plates.

\subsubsection{Figure placement}
\label{sec:figure-placement}

If you are having problems with a floating figure appearing where it
is not wanted, then using \texttt{[!t]} as the optional argument to
the \texttt{figure} environment usually does the trick. The \texttt{t}
asks \LaTeX\ to put the figure at the top of a page, while the
\texttt{!} overrides any qualms it may have about minimum text
fractions or other such niceties.

If all else fails, you can ditch the \texttt{figure} environment
entirely and place it by hand. In this case, you must use
|\figcaption| instead of |\caption|. The same can be done with tables
(using |\tabcaption|). Obviously, it's not a good idea to do this
until you have virtually finished editing the paper. 

\subsubsection{Recalcitrant PostScript files}
\label{sec:postscript-files}

Many authors will send you totally broken PostScript files. Examples
include figures that stubbornly refuse to be re-sized, or that always
appear in the bottom right corner of the page, or that mysteriously
rub out bits of the surrounding text. Prime offenders seem to be
SuperMongo and sundry MS-Windows and Mac programs. 

In many cases, the problem is due to the lack of a proper BoundingBox
(\LaTeX\ needs this in order to know the size of the figure). Look for
a line like 
\begin{Code}
  |%%BoundingBox: 0 0 612 792|
\end{Code}
near the top of the PostScript file. If the file doesn't have one,
then you will have to add one. The easiest way to do this is with
\texttt{gs} (Ghostscript). If you type something like `\texttt{gs
  -sDEVICE=bbox -q -dNOPAUSE example.ps}', you should get something
like
\begin{Code}
  |%%BoundingBox: 97 231 514 561| \\
  |%%HiResBoundingBox: 97.999995 231.000|\dots\\
  |GS>| 
\end{Code}
Just type `\texttt{quit}' at the |GS>| and then paste the
|%%BoundingBox| line into the top of the file (in this case,
\texttt{example.ps}). With luck, your problems will then disappear.

In other cases, however, the problem lies with some of the PostScript
operators used in the file. Somewhere I remember seeing a list of
operators that should not be used in ``encapsulated''\footnote{That
  is, files intended to be embedded inside another document, rather
  than just being sent straight to a printer.} PostScript files. An
example is the command \texttt{initmatrix}, beloved of SuperMongo.
Sometimes, simply removing the offending command from the file is
sufficient. At other times, more drastic measures are required, such
as the use of \texttt{pstoedit} to convert the file to the FIG format.
The resulting \texttt{.fig} file can then be edited with \texttt{xfig}
and subsequently re-saved as another PostScript file. This can produce
miraculous results in many (but not all) cases\footnote{Sometimes you
  will have to adjust the line widths with \texttt{xfig}}. 

If none of the above work, then you can always print out the figure
and rescan it as a bitmap. To get good results with this, print out
the figure at the largest size you can manage. Then, carefully select
the magnification on your scanner\footnote{Make sure you scan it as a
  1-bit image (B\&W only, no color, no grays).} so that the resolution
of the scanned bitmap will match that of your printer (or an integer
divisor thereof)\footnote{300dpi is a reasonable compromise between
  image quality and file size.} \emph{after it has been scaled to its
  desired size in the document}.\footnote{This sounds a lot more
  complicated than it really is.}

\subsubsection{Plates}
\label{sec:plates}
First, ask yourself if you really need them. If you do, then you can
use the |\plate| command in the following way: 
\begin{Code}
      |\plate{|\\
      |  \includegraphics[width=\textwidth]|\\
      |                  {example.ps}|\\
      |  \figcaption{Caption to plate figure|\\
      |     \label{fig:example} } }|
\end{Code}
Note that you do \emph{not} use the \texttt{figure} environment inside
the argument of the |\plate| command. As a result, you must use
|\figcaption| instead of |\caption|. It is possible to include
multiple figures in a single plate (if they will fit), in which case
you should specify the number of distinct figures, i.e.\ |\figcaption|
commands, as an optional first argument to |\plate|\footnote{This is
  so that the macros can increment the \texttt{figure} counter
  accordingly. The plate itself is not output until the end of the
  document.}, for example:
\begin{Code}
  |\plate[2]{|\\
  |  \includegraphics[width=\textwidth]|\\
  |                  {example1.ps}|\\
  |  \figcaption{Caption to first figure|\\
  |     \label{fig:example1} } }|\\
  |  \vfil|\\
  |  \includegraphics[width=\textwidth]|\\
  |                  {example2.ps}|\\
  |  \figcaption{Caption to second figure|\\
  |     \label{fig:example2} } }|
\end{Code}
You can freely mix |\plate| commands and \texttt{figure} environments
in a document and the figure numbering should always come out right
(so long as you use the optional argument discussed above). The only
restriction is that if there are multiple figures in one plate, then
these will be numbered consecutively (e.g., you can't have Figures~4
and 6 on a single plate page with Figure~5 appearing normally in the
text).

The plates are actually formatted by the |\outputplates| command,
which should appear at the very end of the article, just before the
|\end{document}|. 

\subsection{Balancing the columns on the last page}
\label{sec:Balanc-final-columns}
In the two-column style, the columns on the last page should be as
close to the same length as possible. Making \LaTeX\ do this
automatically is beyond my capabilities, so authors are requested to
balance the columns manually by using the |\adjustfinalcols|
command. This command should be inserted at the place in the text
where the left-hand column should break to give two columns of equal
length. Obviously, this should be left until the final draft of the
document, or left to the editors even.

\subsubsection{Full addresses and column balancing}
\label{sec:full-addr-column}
\textit{This section applies to the main journal only. Conference
  proceedings should not have the full addresses at the end of
  articles.}

The |\adjustfinalcols| command also tries to typeset the authors' full
addresses (previously specified by the |\fulladdresses| command, see
%% \ref{sec:Author-Commands}
\S~3.3 of the ``Author Guide'') at the bottom of the final page. If this
is not possible because of lack of space, then the addresses are put
on the next page. Please note that the mechanism to achieve this is
rather fiddly and, although I have tested it extensively, I can't
guarantee it will work in all conceivable circumstances. Please let me
know if you run across any problems. It does not do a very good job if
there are footnotes on the final page (but is this likely?).  If the
|\adjustfinalcols| command is not used then the full addresses will be
set after the end of the \texttt{thebibliography} environment (on the
same page unless there is no room or we are in the second column of
double-column mode).  If you don't have a bibliography for whatever
reason and you are not using |\adjustfinalcols|, you can explicitly
trigger the full addresses with the |\outputfulladdresses| command.


\section{Putting them all together}
\label{sec:putting-them-all}

Now you are ready to make a single \LaTeX\ document of the entire
volume. First, I recommend copying the final version of all the
individual articles (\texttt{.tex} file plus and \texttt{.ps} figures)
to a clean directory. Now, create a new \texttt{.tex} file, which will
be the master file for the entire volume. In what follows, I will
assume it is called \texttt{master.tex}, but you can call it what you
like\footnote{Maybe something like \texttt{RMAA25-2.tex} for \RMAA,
  vol.~25, num.~2.}. In this file, you should use the \texttt{book}
option to the |\documentclass| command, e.g.:
\begin{Code}
  |\documentclass[proceedings,book]{rmaa}|
\end{Code}


\subsection[What to put in the preamble]{What to put in the
  preamble\footnote{That is, before the
    \textbackslash\texttt{begin\{document\}}.}}
\label{sec:what-put-preamble}

Eventually\footnote{Ignore this paragraph on your first reading --- it
  will make more sense later.}, you may need some |\usepackage|
commands here if optional packages have been loaded by individual
authors. Look for lines like
\begin{Code}
  \texttt{Class rmaa Warning: Package floatflt was requested by
    rksmith  on input line 3.} 
\end{Code}
in the output when you \LaTeX\ \texttt{main.tex}. In the example
given, you would need to add |\usepackage{floatflt}|. The reason for
this is that |\usepackage| is one of the commands that is disabled
within individual articles in the \texttt{book} style, since it is
illegal to try and load the same package twice.

You will probably want the following commands to set up the details of
the volume: 
\begin{Code}
  |% publication details of volume                                     |\\
  |\SetVolume{10}                                                      |\\
  |\SetYear{2000}                                                      |\\
  |\SetMes{junio}                                                      |
\end{Code}
and if it is a \RMAASC\ volume you may want these:
\begin{Code}
  |% details of meeting                                                |\\
  |\SetReunion{M\'exico, D.~F.}|\\
  |           {octubre 25--29}                        |\\
  |\SetEditors{S.~Jane~Arthur,| \dots |}|
\end{Code}
Finally, you want
\begin{Code}
  |\makeindex                                                          |
\end{Code}
so that the index will be generated.


\subsection[What to put in the main body]{What to put in the main
  body\footnote{After the \textbackslash\texttt{begin\{document\}}}} 
\label{sec:what-put-main}
First off, you do \emph{not} want a |\maketitle| command. 


\subsubsection{Front matter}

Second, you should put
\begin{Code}
  |\frontmatter| 
\end{Code}
to begin the front matter (this turns on roman page numbering and a
few other necessary things). Then, you put
\begin{Code}
  |\tableofcontents|
\end{Code}
which will make the ``Table of Contents'' magically appear when it is
ready. 

Next, you can put all the other material that you want to have before
starting the articles proper. This can include (but need not be
limited to) a Preface, a List of Participants, a Group Photograph,
etc. All these can be formatted using the \texttt{frontsection}
environment, which has the following syntax: 
\begin{Code}
  |\begin{frontsection}[|\rmArg{shorttitle}|]{|\rmArg{title}|}{%|
      \dots\\
      \rmArg{commands}\\
      \dots |}| \\
    \dots \\
  |\end{frontsection}|
\end{Code}

\subsubsection{Main matter}

Include individual articles. Optionally separate them with |\part|
commands. 

\subsubsection{Short abstracts}

\begin{Code}
  |\begin{abstracts}|\\
  |   \title{|\dots|}|\\
  |   \author{|\dots|}|\\
  |   \listofauthors{|\dots|}|\\
  |   \indexauthor{|\dots|}|\\
  |   \abstract{|\dots|}|\\
  \dots\\
  |\end{abstracts}|
\end{Code}

\subsubsection{The index}

Explain how it is done. 

\subsubsection{The plates}
\label{sec:plates-1}
The \texttt{book} option disables the |\outputplates| command in the
individual articles, so you must put this command at the end of the
master file if any of the included articles contain plates. The plates
will then all appear at the end of the book. 

A further point to remember is that any macros defined by the author
in his/her article will have been ``forgotten'' by the time the plates
are being output\footnote{I have made all such user macros local to
  their article so that there is no problem with ``contaminating''
  subsequent articles (e.g., Author~A redefines
  \textbackslash\texttt{alpha} to be something weird, while Author~B
  is still expecting it to be $\alpha$).}. Hence, if any macros are
used in the caption text of the plate figures, then the definitions of
these must be repeated inside the argument of the |\plate|
command\footnote{And better use \textbackslash\texttt{def} rather than
  \textbackslash(\texttt{re})\texttt{newcommand}.}.


\section{Glossary}
\label{sec:Gloss}


\begin{description}
\item[Auxiliary Files] 
\item[Bounding Box] 
\item[Class] See \textbf{Document Class}. 
\item[Document Class] A self-contained set of \LaTeX\ macros that
  specify the markup for a certain type of document. These are found
  in files with the \texttt{.cls} suffix, and can be used as the
  argument of a |\documentclass| command. Examples include
  \texttt{rmaa}, \texttt{article}, and \texttt{letter}.
  Compare \textbf{Package}.
\item[Encapsulated PostScript] 
\item[Float] 
\item[\LaTeXe] 
\item[Main Body] 
\item[Package] A set of \LaTeX\ macros to acheive a particular task,
  that can be used with many different \textbf{Document Classes} (qv).
  These are found in files with the \texttt{.sty} suffix, and can be
  used as the argument of a |\usepackage| command. Examples include
  \texttt{inputenc}, \texttt{babel}, \texttt{color}.   
\item[Preamble] The \LaTeX\ commands that come between the
  |\documentclass| command and |\begin{document}|. These commands
  should not produce any printed output. 
\item[Style File] 
\end{description}

\acknowledgments

\paragraph{Acknowledgments:} Many of the comments in \S~\ref{sec:individual-papers} were inspired
by the experiences of Jane Arthur while editing the ``Astrophysical
Plasmas 1999'' proceedings. I am also very gratful to Jane for testing (to
destruction) early versions of the macros and to Luis Felipe Rodr\'\i
guez and Elisabeth Themsel for reporting problems and mistakes and for
advice on editorial style..
\glossary{hello}
\end{document}
