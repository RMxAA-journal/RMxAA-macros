\documentclass[debug,guide]{rmaa}
\usepackage[latin1]{inputenc}
\usepackage[pdftex, debug]{hyperref}

      
\title{The \RMAA\ \LaTeX\ Macros: A Guide for Authors\altaffilmark{1}}
\author{W. J. Henney\altaffilmark{2}
  \affil{CRyA-UNAM, Morelia, Mexico} 
  }
\altaffiltext{1}{\protect\raggedright The latest version of the macros 
  should be available from
  \url{http://www.astrosmo.unam.mx/rmaa}} 
\altaffiltext{2}{\url{mailto:will@astrosmo.unam.mx}}
\fulladdresses{
\item W. J. Henney: Centro de Radioastronom�a y Astrof�sica, UNAM Apdo
  Postal 72-3, 58090 Morelia, Michoac\'an, M\'exico
  (\url{mailto:w.henney@astrosmo.unam.mx})}
    
\shortauthor{Henney} \shorttitle{\RMAA\ Author Guide}

\SetYear{2007} 

\resumen{Este documento describe el uso del ``\LaTeX\ document class''
  \texttt{rmaa.cls}, lo cu\'al est\'a dise\~nado para preparar
  art\'{\i}culos para publicaci\'on en la \emph{\RMAAlong}. }

\abstract{This document describes the use of the \LaTeX\ document
  class \texttt{rmaa.cls}, designed for the preparation of papers to
  be published in the main journal \emph{\RMAAlong}.} 

\begin{document}
\maketitle
\tableofcontents

\section{Introduction}
\label{sec:General}

The version of the \texttt{rmaa} document class described in this User
Guide is \rmVersion{} (\rmDate). Its use requires a relatively recent
version of \LaTeX\ (post-1994). Some familiarity with the use of
\LaTeX\ is assumed in this guide. For the author requiring a general
introduction to \LaTeX, there are several books
\citep[e.g.][]{Latex,Comp} available, as well as many online tutorials
(see list at
\url{http://www.tex.ac.uk/cgi-bin/texfaq2html?label=tutorials*}). This
guide aims to cover basic usage of the macros from an author's point
of view; more advanced topics are discussed in the ``Guide for
Editors'' \citep{EdGuide}.

Note that the PDF version of this document contains clickable links,
both for internal cross-references (e.g., in the Table of Contents)
and for external URLs.  

\section{Download and Installation}
\label{sec:downl-inst}
The latest version of the macros can always be obtained from
\url{http://www.astrosmo.unam.mx/rmaa}. There are three different
packages, each designed for a different audience: authors of main
journal papers (e.g., \texttt{rmaa\rmVersion{}.tar.gz}), authors of
contributions to conference proceedings (e.g.,
\texttt{rmsc\rmVersion{}.tar.gz}), and editors of the main journal or
conference proceedings (e.g.,
\texttt{rmaa\_editor\rmVersion{}.tar.gz}). Please make sure you have
the right package. All packages are also provided in \texttt{.zip}
format, chiefly for the convenience of MS Windows users.

Once you have downloaded the relevant package archive file, move it to
the directory (folder) where you wish to work and unpack the
archive. The procedure to do this may vary between systems. From the
command line on Unix, Linux, or Mac OS X, you can use something like
\texttt{tar -zxf rmaa\rmVersion{}.tar.gz}. 

In order to test that everything is working, you should compile the
example document \texttt{rm-journal-example.tex}. You will need to invoke
\LaTeX{} twice to get the cross references right. The procedure to
view and print the document will vary between systems. On Unix, Linux,
or Mac OS X, you can use \texttt{latex} and \texttt{dvips} to generate
a postscript file (which can be converted to PDF via \texttt{ps2pdf}),
or alternatively you can use \texttt{pdflatex} to generate a PDF file
directly. Note that in the first case, any included graphics must be
in EPS format, while in the second case, they must be in PDF, PNG, or
JPG format. It is recommended that you omit the file extension from
the filenames in the |\includegraphics| command, so that both
\texttt{latex} and \texttt{pdflatex} can automatically find the
correct figure file.

Assuming that the example document compiled and printed correctly, you
can now start writing your own paper, using the example document as a
template. 

\section{The preamble}
\label{sec:preamble}

The first line to appear in your document should be
\begin{Code}
  |\documentclass{rmaa}|
\end{Code}
or
\begin{Code}
  |\documentclass[|\rmArg{optionlist}|]{rmaa}|
\end{Code}
which sets up the document to use the \texttt{rmaa} class, using the
default \texttt{manuscript} option, which is designed for use by
authors submitting papers to the \RMAA{} main journal.  The following
commands can be used after the |\documentclass| command, but before
the |\begin{document}|.

\subsection{The title page}
\label{sec:Author-Commands}

\begin{Code}
  |\title{| \rmArg{title text} |}|
\end{Code}
The |\title| command defines the title of the article. The
title text should be entered in mixed case. It will be
automatically converted to upper case when typesetting the title page,
but remains in mixed case for the index. 
\begin{Code}
  \begin{tabbing}
    |\author{| \= \rmArg{name\#1} |and|  \rmArg{name\#2} \\
      \>|\affil{| \rmArg{address} |} }|
  \end{tabbing}
\end{Code}
The |\author| command defines the authors of the article. Within this
command, one can use the |\affil| command to define the authors'
affiliation. This will be typeset below the authors' names.
The individual authors should be entered in the style |A.~B.~Lastname|
to avoid line breaks within the name. Line breaks may be inserted by
hand using |\\|.  If the authors have various different affiliations,
then one should use the alternative form:
\begin{Code}
  \begin{tabbing}
    |\altaffiltext{1}{|\rmArg{address\#1}|}| \\
    |\altaffiltext{2}{|\rmArg{address\#2}|}| \\
    $\cdots$ \\
    |\author{| \= \rmArg{name\#1} |\altaffilmark{| \rmArg{list\#1} |}| \\
      \> \rmArg{name\#2} |\altaffilmark{| \rmArg{list\#2} |} }| \\
  \end{tabbing}
\end{Code}
Each separate affiliation is specified with the |\altaffiltext|
command and will be typeset in a footnote.\footnote{Each address
  should end with a period.} The first argument to
|\altaffiltext| is the number of the footnote and the second argument
is the text of the affiliation.  The |\altaffilmark| command is then
used within the |\author| command to specify a list of footnoted
affiliations for each author, e.g.\ |\altaffilmark{1,3,4}|. Care
should be taken that the |\altaffiltext| commands are entered in the
correct order, starting with number~1.  It is also possible to combine the
|\affil| and |\altaffilmark| commands.  See the sample document for a more
complicated example.

The affiliations entered via the |\affil| and |\altaffiltext| commands
should be as brief as possible. Full postal and email addresses are
typeset at the end of the article by means of the |\fulladdresses|
command (see below).  

\begin{Code}
  |\resumen{| \rmArg{Spanish text} |}| \\
  |\abstract{| \rmArg{English text} |}| 
\end{Code}
These two commands define the abstract in Spanish and English. The
abstract text may contain several paragraphs, but it should not be
overly long, since both abstracts must fit on the first page unless
you are using the \texttt{manuscript} option.
%\footnote{If they don't, then dire things may happen in the
%  current version.}  
If the author does not use the |\resumen|
command, the Spanish text will be set equal to the English text.
\begin{Code}
  |\setkeyword{| \rmArg{keyword text} |}| 
\end{Code}
The |\setkeyword| command adds \rmArg{keyword text} to the list of
keywords (or subject headings). The keywords should be entered in
alphabetical order in the following style and should be chosen from
the
list\footnote{\url{http://www.journals.uchicago.edu/ApJ/keywords.html}}
used by ApJ, AJ, A\&A, and MNRAS (please do not invent new
keywords!). No more than six keywords are allowed. 


\subsection{Running headers}
\label{sec:running-headers}
\begin{Code}
  |\shortauthor{| \rmArg{name} |}| \\ |\shorttitle{| \rmArg{title} |}|
\end{Code}
These define a shortened version of the author and title for use in
the running headers. Only the author surnames should be used here
(first author et~al.\ if there are more than 3) and ``|\&|'' should be
used instead of ``|and|''. 


\subsection{Postal addresses}
\label{sec:postal-addresses}

The affiliations entered with the |\affil| and |\altaffiltext|
commands should be quite short --- the full postal addresses of the
authors should be entered with the |\fulladdresses| command:
\begin{Code}
  |\fulladdresses{| \\
  |   \item| \rmArg{namelist\#1}|:| \rmArg{address\#1}
  |(|\rmArg{email\#1}|)| \\
  |   \item| \rmArg{namelist\#2}|:| \rmArg{address\#2}
  |(|\rmArg{email\#2}|)| \\
  |   | $\cdots$  |}| 
\end{Code}
An example of the use of this command can be found in the sample
document.  These addresses will be typeset at the end of the article.
Note that each address should end with a period.



\subsection{Indexing}
\label{sec:indexing}

The following commands are used in constructing the table of contents
and index when the \texttt{book} style option is used to typeset an
entire volume:
\begin{Code}
  |\listofauthors{| \rmArg{namelist} |}| \\
  |\indexauthor{| \rmArg{author\#1} |}| \\
  $\cdots$\\
  |\indexauthor{| \rmArg{author\#N} |}|
\end{Code}
where \rmArg{namelist} is in the format |A.~N.~Other,| |B.~Second,|
|\& C.~Third|, while \rmArg{author\#1} \dots \rmArg{author\#N} are in
the format \texttt{Other, A.~N.}  There should be one |\indexauthor|
command for each author and it is important that these appear
\emph{after} the |\listofauthors| and |\title| commands. Although
these commands produce no printed output when working on an individual
paper, it would save the editors some work if you put them in.


\subsection{Other commands for editors}
\label{sec:edit}

The following commands are designed principally for the use of the
editors, rather than authors. 
\begin{Code}
  |\SetVolume{| \rmArg{volume \#} |}| \\
  |\SetFirstPage{| \rmArg{page \#} |}| \\
  |\SetYear{| \rmArg{year} |}| \\
  |\ReceivedDate{| \rmArg{date} |}| \\
  |\AcceptedDate{| \rmArg{date} |}| \\
  |\SetMSnumber{| \rmArg{manuscript \#} |}| 
\end{Code}
These should all be self-explanatory. The first three commands set
quantities that are used in constructing the header to the first page.
The page range for the article is calculated automatically. This means
that \LaTeX\ must be run twice after changing the first page with
|\SetFirstPage|. The |\SetMSnumber| command is for use with the
\texttt{manuscript} option. 


\section{The Main Body}
\label{sec:main-text}

The main body of the document should be enclosed within the following
pair of commands: 
\begin{Code}
  |\begin{document}| \\
    \dots \\
    \rmArg{article text} \\
    \dots \\
    |\end{document}|
\end{Code}
The first command after the |\begin{document}| should be 
\begin{Code}
  |\maketitle|
\end{Code}
This will format title, authors, abstracts and keywords. 

Within the main body of the document all standard \LaTeX\ commands can
be used. Commands provided by the many optional packages distributed
with \LaTeX\ may also be used so long as the package is loaded via the 
|\usepackage| command in the preamble. However, authors are requested
to avoid using commands that change the document fonts, page layout or 
other ``stylistic'' parameters. One should also bear in mind that not
all optional packages have been checked for compatibility with the
\texttt{rmaa} class. 


\subsection{Sectioning commands and cross references}
\label{sec:Sectioning-commands}
Authors are encouraged to use the standard \LaTeX\ sectioning commands 
to subdivide their article:
\begin{Code}
  |\section{| \rmArg{Title} |}| \\
  |\subsection{| \rmArg{Title} |}| \\
  |\subsubsection{| \rmArg{Title} |}| \\
  |\paragraph{| \rmArg{Title} |}|
\end{Code}
These will be automatically typeset in the RevMexAA style.
Cross-referencing is made easier by the use of the
|\label{|\rmArg{label}|}| command immediately after each sectioning
command, where \rmArg{label} is any mnemonic string.  Elsewhere in the
document, the section can then be referred to as
|\S~\ref{|\rmArg{label}|}|. The |\label| command can also be used with
equations and with figures and tables (see below). The style that
should be used for cross-references is, for example, Figure~3,
Table~1, equation~(12), and \S~5.1, where the section symbol ``\S'' is
produced by the \LaTeX\ command ``|\S|''.

\subsection{Tables}
\label{sec:Tables}
Tables should be entered using the \texttt{table} and \texttt{tabular}
environments in the following way:
\begin{Code}
  |\begin{table}|\\
    |\caption{|\rmArg{Table Heading}| \label{|\rmArg{label}|} }|\\
    |\begin{tabular}{|\rmArg{Format}|}\toprule|\\
      \rmArg{Header Line}| \\ \midrule|\\
      \rmArg{First Data Line}| \\|\\
      \dots\\
      \rmArg{Last Data Line}| \\ \bottomrule|\\
    |\end{tabular}| \\
  |\end{table}| \\
\end{Code}
In two-column styles, if it is required that the table span both
columns, then |\begin{table*}| \dots |\end{table*}| should be used
instead. Details of more advanced techniques, together with advice on
producing high-quality tables is given in \citet{booktabs}. 

Footnotes to the table caption, to column headings, or to individual
entries can be specified by the |\tabnotemark| and |\tabnotetext|
commands. An example of their use is given in the sample document
\texttt{rm-journal-example.tex}. 

Experimental support for multi-page tables is provided via the
\texttt{longtable} environment. Advice on its use will be provided on
request. 
%% TODO - WJH 10 Sep 2007 - document longtable usage

\subsection{Figures}
\label{sec:Figures}
Figures should be included using the standard \LaTeX\ figure
environment. Authors are encouraged to embed postscript figures
directly in their document, for which we strongly recommend the use of
the standard \LaTeX\ command
|\includegraphics|.\footnote{Historically, many different macro
  packages for the inclusion of postscript figures have been in use
  because older versions of \LaTeX\ lacked a standard interface (e.g.
  \texttt{epsf}, \texttt{psfig}). There is no longer any reason to use
  these.} The extended features of the |\includegraphics| command
provided by the \texttt{graphicx} package may be used. There is no
need to explicitly load this package, it is loaded automatically.

For a single-column figure, the following commands
could be used: 
\begin{Code}
  |\begin{figure}\centering|\\
      |\includegraphics{|\rmArg{psfilename}|}|\\
  |\caption{|\rmArg{fig caption}| \label{|\rmArg{label}|} }| \\
  |\end{figure}|
\end{Code}
By default, the graphics file will be scaled to the full column
width. Alternatively, one of the optional arguments \texttt{width},
\texttt{height}, or \texttt{scale} can be used to set the desired
size. Further examples are given in \texttt{rm-journal-example.tex}.  For
figures that span two columns, one should use \texttt{figure*} instead
of \texttt{figure}.  For further information, authors are encouraged
to consult \citet{Graph}, which contains a wealth of detail on the use
of imported graphics.  More details on the preparation of figures is
also given in the ``Guide for Editors'', q.v.



% Plates aren't used any more - WJH 10 Sep 2007
% \subsection{Plates}
% \label{sec:plates}

% If you want any of your figures to appear as glossy plates, then you
% should inform the editors of the journal or conference proceedings. The
% |\plate| command that is used for formatting these is decribed in the
% ``Guide for Editors''.  

\subsection{Reference list and citations}
\label{sec:Reference-List}


There are three broad methods for dealing with citations.

\subsubsection{Semi-automatic method}
\label{sec:semi-autom-meth}
This is currently the recommended method for most authors. The
reference list at the end of the article is constructed by hand
(preferably using entries generated by ADS), whereas the in-text
citations are handled automatically using commands from the
\texttt{natbib} package.\footnote{The \texttt{natbib} package is
  loaded and configured automatically by \texttt{rmaa.cls}.}

\paragraph{How to enter the reference list}

The reference list should appear at the end of the article and should
be formatted using the standard \LaTeX\ \texttt{bibliography}
environment as follows:
\begin{Code}
  |\begin{thebibliography}|\\
    |\bibitem[|\rmArg{authoryear}|]{|\rmArg{key}|} |\rmArg{text} \\
    \dots \\
  |\end{thebibliography}|
\end{Code}
The required format of the |\bibitem|s is best explained by examples:
\begin{verbatim}
\bibitem[Garc{\'{\i}}a-D{\'{\i}}az \&
  Henney(2007)] {2007AJ....133..952G}
  Garc{\'{\i}}a-D{\'{\i}}az, M.~T., \&
  Henney, W.~J.\ 2007, \aj, 133, 952

\bibitem[Henney et al.(2007)]
  {2007AJ....133.2192H} Henney, W.~J.,
  O'Dell, C.~R., Zapata, L.~A.,
  Garc{\'{\i}}a-D{\'{\i}}az, M.~T.,
  Rodr{\'{\i}}guez, L.~F., \& Robberto,
  M.\ 2007, \aj, 133, 2192

\bibitem[Henney(2007)]
  {2007dmsf.book..103H} Henney, W.~J.\
  2007, in Diffuse Matter from Star Forming
  Regions to Active Galaxies: A Volume
  Honouring John Dyson, Edited by
  T.~W. Hartquist, J.~M. Pittard, and
  S.~A.~E.~G. Falle. Series:
  Astrophysics and Space Science
  Proceedings. ISBN-10 1-4020-5424-6
  (Dordrecht: Springer), p.103
\end{verbatim}
By far the easiest way of obtaining the |\bibitem|s is by using
ADS. Select the references you need by ticking the boxes in the
``Query Results from the ADS Database'' list. Then go down to the
section that says ``Retrieve the above records in other formats or
sort order'' and choose ``AASTeX format'' from the first drop-down
list. If you have selected many references at once, then you may also
want to choose ``Sort by first author name''. The resultant
|\bibitem|s can then be cut-and-pasted into your paper. In the case of
references to papers in common astronomy journals, the ADS results are
nearly always fine. For references to books and conference
proceedings, the quality is very variable and the diligent author may
need to edit the result to conform to the standard style (see recent
issues of \RMAA{} or \RMAASC{} for examples).
 
If you have long FTP or HTTP addresses in your reference list (or
anywhere else), then you should use the |\url| command, which avoids
problems with special characters in the address (such as |~|) and
allows \LaTeX{} to choose sensible linebreaks:
\begin{Code}
  |\url{http://www.astroscu.unam.mx/~rmaa}|
\end{Code}

\paragraph{How to cite references in the text}

For a citation used as a noun in a normal sentence, one should use
|\citet{|\rmArg{key}, \rmArg{key}, \dots|}|, where each \rmArg{key}
matches that of a |\bibitem| in the \texttt{thebibliography}
environment:
\begin{Code}
\begin{verbatim}
\dots following the excellent suggestion 
of \citet{2007AJ....134.1679O}. 
\end{verbatim}
  \itshape
  \dots following the excellent suggestion of
  \citet{2007AJ....134.1679O}. 
\end{Code}
For a parenthetical citation, one should use |\citep{|\rmArg{key},
  \rmArg{key}, \dots|}|:
\begin{Code}
\begin{verbatim}
The subject of photoionized stellar jets 
\citep{2007AJ....133..952G, 
2007AJ....133.2192H} is a fascinating 
one.
\end{verbatim}
  \itshape
  The subject of photoionized stellar jets \citep{2007AJ....133..952G,
    2007AJ....133.2192H} is a fascinating one.
\end{Code}
A useful quick-reference sheet for the \texttt{natbib} commands is
\citet{natnotes}, which explains more complicated techniques, such as
in the following example:
\begin{Code}
\begin{verbatim}
However,
\citeauthor{2007AJ....133.2192H}'s
model fails to consider (pace
\citealp{2007AJ....133.2343O}) the
incontrovertible argument of
\citet[eq.~13]{2007A&A...471..193S},

which renders futile any such effort
\citep[see also][]{2007AJ....134.1679O} 
and which should already have been well 
known in \citeyear{2007AJ....133.2192H}.
\end{verbatim}
  \itshape
  However, \citeauthor{2007AJ....133.2192H}'s model fails to consider
  (pace \citealp{2007AJ....133.2343O}) the incontrovertible argument
  of \citet[eq.~13]{2007A&A...471..193S}, which renders futile any
  such effort \citep[see also][]{2007AJ....134.1679O} and which should
  already have been well known in \citeyear{2007AJ....133.2192H}.
\end{Code}


\subsubsection{Fully automatic method}
\label{sec:fully-autom-meth}
This method is currently experimental, although it seems to work
fine. The in-text citations are specified using \texttt{natbib}
commands in exactly the same way as in the previous method, but now
\BibTeX{} is used to automatically generate the reference list
too. The big advantage of this method is that it guarantees
consistency between the in-text citations and the reference list and
ensures that the reference list is in the correct order.

You will need one or more \texttt{.bib} files that contain all your
references in \BibTeX{} format. Instead of using the
\texttt{thebibliography} environment, you simply put 
\begin{Code}
  |\bibliography{|\rmArg{bibfile}, \rmArg{bibfile}, \dots|}|
\end{Code}
at the end of your document, where each \rmArg{bibfile} is the name of
a \texttt{.bib} file (but without the \texttt{.bib} extension). There
is no problem if the \rmArg{bibfile}s contain many more references
than you will use---\BibTeX{} will just select the ones that you
cite. 

One very easy way to create a useful \texttt{.bib} file is as
follows. Create a ``Private Library'' in ADS for your project and
populate it with every paper that might conceivably be relevant. This can
be done in various stages, starting of with relevant keyword, object,
or author searches. Once you have saved a few core papers, you can
``grow'' the library by retrieving all the references in or citations
to the papers already in the library. Once you have a few hundreds (or
thousands) of papers in the library, then you can dump it all in
\BibTeX{} format by pressing the ``Select all records'' button,
choosing ``BIBTEX reference list'' as the return format, and then
pressing the ``Retrieve selected records'' button.

%% TODO - WJH 10 Sep 2007 - Add description of RefTeX mode here someday

Using this method requires an extra step when compiling your
document. After adding new citations you have to run \LaTeX{},
\BibTeX{}, then \LaTeX{} again. 

\subsubsection{Manual method}
\label{sec:manual-method}

Of course, it is also possible to enter both the reference list and
the in-text citations by hand. However, this is tedious and
error-prone, so it is not recommended. 

\subsection{Special commands}
\label{sec:Special-commands}
For the convenience of authors, special commands 
for astronomical
journals and some common 
astronomical and mathematical symbols are
provided. These are the same as those provided by the AAS macros
\citep{AAS}. 

\subsection{Appendices}
\label{sec:appendices}

If you have appendices to your article, you can use something like the
following: 
\begin{Code}
\begin{verbatim}
\begin{appendices}

  \section{First Appendix}
  \label{sec:ap-A}

  Text of first appendix.

  \section{Second Appendix}
  \label{sec:ap-B}

  Text of second appendix.

\end{appendices}
\end{verbatim}
\end{Code}
This goes after the acknowledgments but before the
bibliography. Equations in appendices will be labeled A1, A2, B1, B2,
etc.

\adjustfinalcols

\begin{thebibliography}


\bibitem[American Mathematical Society(2002)]{amsmath} American
  Mathematical Society 2002, ``User's Guide for the amsmath Package
  (Version 2.0)'', available from
  \url{http://www.ctan.org/tex-archive/macros/latex/required/amslatex/math/amsldoc.pdf}

\bibitem[Biemesderfer \& Barnes(1995)]{AAS} Biemesderfer, C. \&
  Barnes, J.\@ 1995, ``The AAS\TeX\ Macros for Manuscript Preparation'', 
  available at \url{http://www.journals.uchicago.edu/AAS/AASTeX/}

\bibitem[Daly(2007)]{natnotes} Daly, P. W.\@ 2007 ``Reference sheet
  for \texttt{natbib} usage'', available from
  \url{http://tug.ctan.org/macros/latex/contrib/natbib/natnotes.pdf}

\bibitem[Fear(2005)]{booktabs} Fear, S. 2005, ``Publication quality
  tables in LATEX'', available from
  \url{http://www.ctan.org/tex-archive/macros/latex/contrib/booktabs/booktabs.pdf}

\bibitem[Mittelbach et al.{}(2004)]{Comp} Mittelbach, F., Goosens, M., 
  Braams, J., Carlisle, D., Rowley, C.\@ 2004, ``The \LaTeX\ Companion
  (2nd Ed)'', (Reading, MA: Addison-Wesley)
\bibitem[Henney(2000)]{EdGuide} Henney, W. J.\@ 2000 ``The \RMAA\
  \LaTeX\ Macros: A Guide for Editors'', available as file
  \texttt{rmeditor.tex} in the |rmaa_editor| distribution
\bibitem[Lamport(1985)]{Latex} Lamport, L.\@ 1985, ``\LaTeX\ --- A
  Document Preparation System --- User's Guide and Reference Manual''
  (Reading, MA: Addison-Wesley) 
\bibitem[Reckdahl(1997)]{Graph} Reckdahl, K.\@ 1997, ``Using Imported
  Graphics in \LaTeXe'', available from
  \url{http://tug.ctan.org/info/epslatex.ps}

\bibitem[O'Dell et al.(2007b)]{2007AJ....134.1679O} O'Dell, C.~R., Sabbadin, 
F., \& Henney, W.~J.\ 2007, \aj, 134, 1679 


\bibitem[Stasi{\'n}ska et al.(2007)]{2007A&A...471..193S} Stasi{\'n}ska, 
G., Tenorio-Tagle, G., Rodr{\'{\i}}guez, M., \& Henney, W.~J.\ 2007, \aap, 
471, 193 


\bibitem[O'Dell et al.(2007a)]{2007AJ....133.2343O} O'Dell, C.~R., Henney, 
W.~J., \& Ferland, G.~J.\ 2007, \aj, 133, 2343 


\bibitem[Henney et al.(2007)]{2007AJ....133.2192H} Henney, W.~J., O'Dell, 
C.~R., Zapata, L.~A., Garc{\'{\i}}a-D{\'{\i}}az, M.~T., Rodr{\'{\i}}guez, 
L.~F., \& Robberto, M.\ 2007, \aj, 133, 2192 


\bibitem[Garc{\'{\i}}a-D{\'{\i}}az \& Henney(2007)]{2007AJ....133..952G} 
Garc{\'{\i}}a-D{\'{\i}}az, M.~T., \& Henney, W.~J.\ 2007, \aj, 133, 952 

%in {\texttt{docs/latex/general/}} of most \LaTeX\ 
%  distributions. 

\end{thebibliography}

\end{document}
